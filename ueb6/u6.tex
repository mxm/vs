\documentclass[a4paper,9pt]{article}
\usepackage[margin=1in]{geometry}

\usepackage[utf8]{inputenc}
\usepackage{ngerman}
\usepackage{fancyhdr}
\usepackage{color}

\renewcommand\thesubsection{\thesection.\alph{subsection}}

\pagestyle{fancy}

\setlength{\headheight}{60pt}
\lhead[]{\Large{\textbf{Verteilte Systeme 2012: 6. \"Ubungszettel}} \\
\quad \\ \large{Schintke, Sch\"utt\\ 07.06.2012 }}
\rhead[]{\large{Max Michels \\ Philipp Borgers \\ Sascha Sch\"onfeld}}

\begin{document}

\section{Konsistenzmodell f\"ur einen Aktienhandel}
F\"ur einen Aktienhandel sollte kausale Konsistenz verwendet werden. Die wichtigste Voraussetzung f\"ur einen Aktienhandel ist, dass die \"Anderungen der Werte (Aktienpreise) stets konsistent sind, die auch kausal voneinander abh\"angen. Wert\"anderungen, die voneinander unabh\"angig sind, sind nicht relevant f\"ur die einzelne Aktie. 

\section{Konsistenzmodelle}
\subsection{}
Zeigen sie, dass der folgende Verlauf nicht kausal konsistent ist:

\begin{tabbing}
$P_{1}$: \= W(a)0 \= \quad \= W(a)1 \\
$P_{2}$: \> R(a)1 \> \quad \> W(b)2 \\
$P_{3}$: \> R(b)2 \> \quad \> R(a)0
\end{tabbing}

Der Verlauf ist nicht kausal konsistent, da $P_{2}$ a = 1 liest, bevor $P_{3}$ a = 0 liest. Die Bedingung f\"ur kausale Konsistenz sagt aus, dass Schreiboperationen, die in Kausalit\"at stehen (Was f\"ur die beiden Operationen in $P_{1}$ zutrifft, da sie die selbe Variable beschreiben), von allen Prozessen in genau der Reihenfolge gesehen werden m\"ussen, in der sie ausgef\"uhrt wurden.

\subsection{}
Ist der Speicher, der der folgenden Ausf\"uhrung zugrundeliegt, sequentiell konsistent (vorausgesetzt, alle Variablen sind zun\"achst auf Null gesetzt)?

\begin{tabbing}
$P_{1}$: \= R(x)1 \= \quad \= R(x)2 \= \quad \= W(y)1 \\
$P_{2}$: \> W(x)1 \> \quad \> R(y)1 \> \quad \> W(x)2
\end{tabbing}

Der Speicher ist nicht sequentiell konsistent. Bei den jeweils zweiten Anweisungen gibt es bereits Probleme: \newline
W(x)1 $\rightarrow$ R(x)1 $\rightarrow$ \textcolor{red}{R(x)2} ist nicht korrekt, da x = 1. \newline
W(x)1 $\rightarrow$ R(x)1 $\rightarrow$ \textcolor{red}{R(y)1} ist ebenfalls falsch, da y noch nicht initialisiert wurde.

\subsection{}
Welchen Konsistenzmodellen entspricht b) ggf. zus\"atzlich? \newline
Die Reihenfolge entspricht lediglich der FIFO-Konsistenz.

\pagebreak{}
\section{}
\subsection{Code}

Wir haben lediglich einen Multicast in Erlang implementiert. Für die Umsetzung des Best-Effort-Mulicasts mit kausaler Ordnung fehlte uns wegen wenig Praxis in Erlang die nötige Vorstellungskraft, um das Verstandene in Code zu gießen.
Zunächst werden fünf Prozesse gestartet, die fortwährend die loop-Funktion ausführen. Die Loop-Funktion wartet auf Nachrichten verschiedene Art: eine join-Nachricht, die das Beitreten einer Gruppe ermöglicht, eine mutlicast-Nachricht zum Senden von Nachrichten an alle Gruppenmitglieder und eine Nachricht zum Ausgeben der empfangegenen Nachrichten.

\begin{verbatim}
-module(multicast).

-compile(export_all).


test() ->
    P1 = start(),
    P2 = start(),
    P3 = start(),
    P4 = start(),
    P5 = start(),
    Group = [P1, P2, P3, P4, P5],
    P1 ! {join, Group},
    P1 ! {multicast, "This is a message"},
    Group.

start() ->
    Pid = spawn(multicast, loop, [{0, 0, []}]),
    Pid.    

loop({S, R, Group}=State) ->
    receive
        {join, Members} ->
            io:format("Process ~p joined group ~p~n", [self(), Members]),
            loop({S, R, Members});
        %% start a multicast message
        {multicast, Message} ->
            io:format("Received multicast init message ~p~n", [Message]),
            io:format("Sending multicast message to group ~p~n", [Group]),
            % send message to each member of the group
            Fun = fun (Pid) ->
                    Pid ! {msg, self(), S, Message}
            end,
            lists:foreach(Fun, Group),
            %% increase seq. number
            loop({S+1, R, Group});
        {msg, Src, S2, Message} ->
            io:format("Process ~p received Message ~p~n", [self(), {Src, S2, Message}]),
            loop(State);
        X ->
            io:format("Fail...~p~n",[X]),
            loop(State)
    end.
\end{verbatim}
\pagebreak{}

\subsection{Testlauf}

\begin{verbatim}
Eshell V5.8  (abort with ^G)
1> c(multicast).
{ok,multicast}
2> multicast:test().
Process <0.39.0> joined group [<0.39.0>,<0.40.0>,<0.41.0>,<0.42.0>,<0.43.0>]
[<0.39.0>,<0.40.0>,<0.41.0>,<0.42.0>,<0.43.0>]
Received multicast init message "This is a message"
Sending multicast message to group [<0.39.0>,<0.40.0>,<0.41.0>,<0.42.0>,
                                    <0.43.0>]
Process <0.39.0> received Message {<0.39.0>,0,"This is a message"}
Process <0.40.0> received Message {<0.39.0>,0,"This is a message"}
Process <0.41.0> received Message {<0.39.0>,0,"This is a message"}
Process <0.42.0> received Message {<0.39.0>,0,"This is a message"}
Process <0.43.0> received Message {<0.39.0>,0,"This is a message"}
\end{verbatim}

\end{document}

